\chapter{Métamathématique}


Ce chapitre a pour objetif de réglementer aussi formellement que possible la pratique mathématique. Évidemment, il nous est impossible d'être aussi rigoureux FINIR, DIRE QU'ON EST DANS LE METALANGAGE. POUR CELA ON SE LIMITE A DES MANIPULATIONS ALGORITHMIQUES SUR DES STRUCTURES FINIES ---> TOUT CE CHAPITRE EST IMPLEMENTABLE SUR UNE MACHINE.

Tous les symboles relatifs au métalangage seront dans la suite typographiés en gras. Typiquement, les symboles $\meta{0}, \meta{1}, \meta{2}, \dots$ désignent les entiers positifs du métalangage.


\section{Syntaxe}

  \begin{discussion}
    On appelle \emph{symbole logique} les signes $\neg, \lor, \liota, \exists$.
  \end{discussion}

  \begin{discussion}
    Un \emph{langage} est la donnée:
    \begin{itemize}
      \item d'une liste finie de \emph{symboles de relations} $\meta{R_0}, \meta{R_2}, \dots$ munis de leur \emph{arité}, contenant le \emph{symbole d'égalité} $=$ d'arité $2$;
      
      \item d'une infinité de \emph{symboles de variables} $\meta{v_0}, \meta{v_1}, \meta{v_2}, \dots$\footnote{Dans notre cas, ce seront des lettres latines et grecques suivies d'un nombre arbitraire de primes, par exemple $x$ ou $\alpha''$.}.
    \end{itemize}
    Dans cette description, il est bien sûr supposé implicitement que les collections des symboles logiques, de relations et de variables sont disjointes. Considérées simultanément, elles forment ce que l'on appelle les \emph{symboles} du langage.
    
    On fixe dans la suite un langage $\meta{\mL}$. Une \emph{expression} de $\meta{\mL}$ est un arbre enraciné dont chaque n\oe{}ud est étiqueté par un symbole de $\meta{\mL}$ ou par une référence à un autre n\oe{}ud. Graphiquement, on représentera les référence d'un n\oe{}ud à un autre par une flèche pointillée. Les expressions seront désignées dans la suite par des lettres grecques.
    
    \textbf{TODO}: Il n'est pas commode de représenter un arbre à chaque fois, dans la pratique courante on les écrira en ligne (?) avec des parenthèses
  \end{discussion}


  

  \begin{discussion}[Substitution]
    Étant données $\meta{\alpha}$ et $\meta{\xi}$ deux expressions et $\meta{x}$ une variable, l'écriture $\meta{\xi}\subst{\meta{x}}{\meta{\alpha}}$ désigne l'expression obtenue en remplaçant dans $\meta{\xi}$ toutes les feuilles étiquetées par $\meta{x}$ par l'arbre $\meta{\alpha}$.
  \end{discussion}

  \begin{discussion}
    Étant donnés $\meta{s}$ un symbole de $\meta{\mL}$ et $\meta{\xi_1}, \dots, \meta{\xi_n}$ des expressions de $\meta{\mL}$, on désigne par $\meta{s}\p{\meta{\xi_1}, \dots, \meta{\xi_n}}$ l'expression (\textbf{TODO:} changer graphique):
    \begin{figure}[H]
      \centering
      \includegraphics{formuleCas1}
    \end{figure}
    \noindent Alors:
    \begin{itemize}
      \item si $\meta{\alpha}$ et $\meta{\beta}$ sont deux formules, on note $\neg\meta{\alpha}$ et $\meta{\alpha} \lor \meta{\beta}$ en lieu et place de $\neg\p{\meta{\alpha}}$ et $\lor\p{\meta{\alpha}, \meta{\beta}}$;
      
      \item si $\meta{x}$ est une variable, $\p{\meta{sx}}\meta{\xi}$ est l'expression obtenue en remplaçant chaque étiquette $\meta{x}$ dans $\meta{s}\p{\meta{\xi}}$ par une reférence à la racine (étiquetée par $\meta{s}$).
    \end{itemize}
  \end{discussion}

  \begin{discussion}
    Les \emph{formules} et les \emph{termes} de $\meta{\mL}$ sont définis récursivement de la façon suivante:
    \begin{itemize}
      \item si $\meta{R}$ est un symbole de relation d'arité $n$ et si $\meta{\tau_1}, \dots, \meta{\tau_n}$ sont des termes, alors $\meta{R}\p{\meta{\tau_1}, \dots, \meta{\tau_n}}$ est une formule (dite \emph{atomique});
      
      \item si $\meta{\phi}$ et $\meta{\psi}$ sont des formules, alors $\neg\meta{\phi}$ et $\meta{\phi} \lor \meta{\psi}$ également;
      
      \item si $\meta{\phi}$ est une formule et $\meta{x}$ une variable, alors $\p{\exists\meta{x}}\meta{\phi}$ est une formule;
    \end{itemize}
    et:
    \begin{itemize}
      \item les variables sont des termes;
      \item si $\meta{\phi}$ est une formule et si $\meta{x}$ est une variable, alors $\p{\liota\meta{x}}\meta{\phi}$ est un terme.
    \end{itemize}
  \end{discussion}

  \begin{observation}
    Si $\meta{\xi}$ est une formule (resp. un terme), $\meta{x}$ une variable et $\meta{\tau}$ un terme, alors $\meta{\xi}\subst{\meta{x}}{\meta{\tau}}$ est une formule (resp. un terme).
  \end{observation}

  \begin{justif}
    Les symboles de variables apparaissent forcément dans les feuilles (\textbf{TODO}).
  \end{justif}


\section{Sémantique}


\section{Relation d'égalité}

