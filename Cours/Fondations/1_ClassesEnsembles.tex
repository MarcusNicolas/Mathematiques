\chapter{Classes et ensembles}


\section{Axiomatique NBG}


  \begin{discussion}
    La méthode axiomatique consiste à faire reposer un édifice mathématique sur un nombre restreint de propositions tenues pour vraies, appelées \emph{axiomes}. Le système axiomatique sur lequel nous nous baserons dans la suite de ce cours se nomme système von Neumann--Bernays--Gödel, ou, de manière plus concise, système NBG.
    
    Les objets dont traite cette théorie sont appelées des \emph{classes}, liées les unes par rapport aux autres par les relations d'égalité et d'appartenance, notées $=$ et $\in$. Le premier axiome sert à lier ces deux symboles.
  \end{discussion}

  \begin{discussion}[Relation d'inclusion]
    dd
  \end{discussion}

  \begin{axiome}[extensionnalité]
    $\pourtout{x} \pourtout{y} x=y \Leftrightarrow \p{x \subseteq y \land y \subseteq x}$
  \end{axiome}

  \begin{axiome}[réunion]
    $\pourtout{x} \existe{y} \pourtout{z} {z \in y} \Leftrightarrow \p{\existe{w} z \in w \land w \in x}$
  \end{axiome}

  \begin{axiome}[puissance]
    $\pourtout{x} \existe{y} \pourtout{z} {z \in y} \Leftrightarrow {z \subseteq x}$
  \end{axiome}

  \begin{axiome}[infini]
    $\existe{I} {I \neq \vide} \land \p{\pourtout{x} {x \in I} \Rightarrow {x \cup \classe{x} \in I}}$
  \end{axiome}

  \begin{axiome}[fondation]
    $\pourtout{x \in \univers} {x \neq \vide} \Rightarrow \p{\existe{y \in x} x \cap y \neq \vide}$
  \end{axiome}



\section{Arithmétique ordinale}

