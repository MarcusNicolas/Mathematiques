\chapter{Métamathématique}

Que signifie l'expression \emph{faire des mathématiques} ? À ma connaissance, il n'existe pas au sein de la communauté mathématique de réponse faisant l'unanimité. Il est donc nécessaire, comme toujours, de prendre les problèmes à la racine et de définir aussi rigoureusement que faire se peut en quoi ce en quoi consiste la pratique mathématique telle qu'envisagée dans ce cours, c'est-à-dire entre autres à donner un sens à des mots comme \emph{énoncé}, \emph{démonstration}, \emph{théorie}, etc.

Le lecteur aura remarqué que nous avons promis dans ce chapitre une exposition \emph{aussi rigoureuse que possible}, et non pas rigoureuse. C'est parce que tout l'objectif de ce chapitre est justement de clarifier ce que nous entendons par le mot \emph{rigueur}, qui sera pour nous un synonyme de \emph{mathématique}. PARLER DE MÉTALANGAGE


\section{Langage et syntaxe}

  \subsection*{Substitutions}

    \begin{ex}
      oui
    \end{ex}


\section{Théories mathématiques}


\section{Relation d'égalité}
