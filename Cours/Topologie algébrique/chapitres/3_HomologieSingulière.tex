\chapter{Homologie singulière}


\section{Objets simpliciaux}

  \begin{déf}
    La catégorie simpliciale $\catSimpliciale$ est la restriction de la catégorie des ensembles ordonnés aux objets $[n] \defeq \classe{0, \dots, n}$ (pour $n \in \N$).
  \end{déf}

  COFACES, ETC

  \begin{ex}
    On a un foncteur naturel $\app{\Delta}{\catSimpliciale \fleche \Top}$, défini par:
    \begin{itemize}
      \item pour tout $n$, $\Delta\p{[n]}$ est le $n$-simplexe standard:
      \[\Delta^n \defeq \conv\p{e_0, \dots, e_n} \subseteq \R^{n+1}\]
      où $\p{e_i}$ désigne la base euclidienne canonique de $\R^{n+1}$.

      \item si $f$ est un morphisme $[n] \fleche [m]$, alors $\Delta\p{f}$ est l'unique application affine $\Delta^n \fleche \Delta^m$ qui envoie chaque $e_i$ sur $e_{f\p{i}}$.
    \end{itemize}
  \end{ex}

  \begin{déf}
    Un objet simplicial $X$ d'une catégorie $\cat{C}$ est un foncteur:
    \[\app{X}{\catSimpliciale^\op \fleche \cat{C}}\]
    Autrement dit, c'est la donnée d'objets $X_n$ de $\cat{C}$ pour tout $n$, et TODO
  \end{déf}

  \begin{discussion}
    À un module simplicial $S$, on associe un complexe de chaînes:
    \begin{figure}[h]
      \centering
      \begin{tikzcd}[arrows=to,column sep=.5cm]
          \cdots \rar["\partial"] & S\p{[n]} \rar["\partial"] & S\p{[n-1]} \rar["\partial"] & \cdots\rar["\partial"] & S\p{[0]} \rar & 0
      \end{tikzcd}
    \end{figure}

    \noindent où l'opérateur bord $\app{\partial}{S\p{[n]} \fleche S\p{[n-1]}}$ est:
    \[\partial \defeq \sum_{i=0}^n \p{-1}^i S\p{d_i}\]
  \end{discussion}


\section{Homologie et cohomologie}

  Soit $\p{B, b_0}$ un espace topologique délaçable pointé.

  \begin{déf}
    Pour $n \geq 1$ et $0 \leq i \leq n$, la $i$-ième face de $\Delta^n$ est l'unique application affine $\app{\partial_i^n}{\Delta^{n-1} \fleche \Delta^n}$ vérifiant $\partial_i^n\p{e_0, \dots, e_{n-1}} = \p{e_0, \dots, \hat{e_i}, \dots, e_n}$.
  \end{déf}

  Soit $X$ un espace topologique muni d'une application $\app{f}{X \fleche B}$.

  \begin{déf}
    Un $n$-simplexe singulier de $X$ est une application $\Delta^n \fleche X$.
  \end{déf}


  \begin{discussion}
    Si $n \geq 0$, le groupe des $n$-chaînes singulières de $X$ $f$-marquées dans $B$ est le groupe abélien libre:
    \[C_n^{(f)}\p{X; B} \defeq \bigoplus_{\p{\sigma, \gamma}} \Z\]
    où la somme porte sur l'ensemble des $n$-simplexes singuliers $\sigma$ de $X$ marqués par un chemin $\gamma$ de $B$ reliant $b_0$ à $f \circ \sigma\p{c_n}$. ($c_n$ est BARYCENTRE) Lorsqu'il n'y a pas d'ambiguïté, on écrit $C_n\p{X; B}$ au lieu de $C_n^{(f)}\p{X; B}$.\\
    
    On définit un opérateur  ($\Z[\pi]$-linéaire)
    \[\app{\partial}{C_n^{(f)}\p{X; B} \lfleche C_{n-1}^{(f)}\p{X; B}}\]
    par:
    \[\partial\p{\sigma, \gamma} = \sum_{i=0}^n \p{-1}^i \p{\sigma \circ \partial_i^n, \gamma \cdot \p{f \circ \sigma \circ \alpha_i}}\]
    où $\alpha_i$ désigne n'importe quel chemin de $\Delta^n$ reliant $c_n$ à $\partial_i^n\p{c_{n-1}}$ (pour tout $0 \leq i \leq n$). On vérifie que $\partial^2 = 0$.
  \end{discussion}


\section{Fonctorialité}

\section{Excision}
